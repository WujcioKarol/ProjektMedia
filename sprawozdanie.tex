\documentclass{article}
\usepackage[T1]{fontenc}
\usepackage[utf8]{inputenc}
\usepackage{polski}
\usepackage{amsmath}
\usepackage{graphicx}
\usepackage{minted}
\usepackage{xcolor}
\usepackage{geometry}
\geometry{a4paper, total={170mm,257mm}, left=20mm, top=20mm}
\usemintedstyle{monokai}

\definecolor{bg}{HTML}{282828}

\setminted{
    linenos=true,
    breaklines=true,
    bgcolor=bg, 
}


\title{Opracowanie modelu obliczeniowego metody Har-Xia-Bertoni}
\author{Karol Slomczyński 272223}
\date{\today}

\begin{document}



\maketitle

\section{Wstęp}
Do wykonania projektu użyłem Python'a w wersji 3.10.12 oraz biblioteki math.


\section{Opracowanie kodu}
\subsection{Zabudowa niska}
Jak możemy zauważyć w kodzie, metoda Har-Xia-Bertoni jest 
zaimplementowana w postaci funkcji zależnie od specyfikacji zabudowy i typu trasy
Poniżej znajduje się kod funkcji dla zabudowy niskiej i tras schodkowych oraz poprzecznych:
\begin{minted}{python3}
    def low(f_G: float , R_k: float, r_h: float, h_t: float, h_bd:float) -> float:
    d_h:float = h_t - h_bd
    path_loss = (139.01 + 42.59*math.log10(f_G))\
                -(14.97 + 4.99*math.log10(f_G))*abs(d_h)*math.log10(1+abs(d_h))\
                + (40.67 - 4.57*abs(d_h)*math.log10(1+abs(d_h)))*math.log10(R_k)
    return path_loss
\end{minted}
Po analize udostępnionego dokumentu zauważyłem, że w przypadku zabudowy niskiej typ trasy w przypadku trasy schodkowej i poprzecznej
jest taka sama, dlatego zdecydowałem się na zaimplementowanie jednej funkcji dla obu przypadków.
\newline
Poniżej znajduje się kod funkcji dla zabudowy niskiej i trasy bocznej:

\begin{minted}{python3}
    def low_sideways(f_G: float , R_k: float, r_h: float, h_t: float, h_bd:float) -> float:
    d_h:float = h_t - h_bd
    path_loss = (127.39+31.63*math.log10(f_G))\
                -(14.97+4.99*math.log10(f_G))*abs(d_h)*math.log10(1+abs(d_h))\
                +(40.67-4.57*abs(d_h)*math.log10(1+abs(d_h)))*math.log10(R_k)
    return path_loss
\end{minted}
Wedle dokumentu podane formuły dają poprawne wyniki, gdzie $f_G$ jest w przedziale $( 0.9; 2 )$ GHz, $\Delta h$ jest w przedziale $(-8;6)$m, $R_k$ w przedziale $(0.05;3)$km
\newline
Aby zniwelować błędy związane z obliczeniami zastosowałem współczynnik korekcji dla wysokości budynków, który wynosi:

\[ {\left(\Delta P_L\right)}_{\Delta h_m} = 20\log{(\Delta h_m/7.8)} \]
W przypadku rozpatrywanym w dostarczonym dokumencie średnia geometryczna wysokości budynków wynosi 7.8m.
\newline
Dokument zakłada że odległości od frontu budynku do środka ulicy wynosi 20 metrów. Przez to współczynnik korekcji wynosi
\[ {\left(\Delta P_L\right)}_{r_h} = 10\log(20/r_h) \]

Funkcja dla wszystkich tras po zastosowaniu korekcji wygląda następująco:
\begin{minted}{python3}
    def low(f_G: float , R_k: float, r_h: float, h_t: float, h_bd:float) -> float:
    d_h:float = h_t - h_bd
    path_loss = (139.01 + 42.59*math.log10(f_G))\
                -(14.97 + 4.99*math.log10(f_G))*abs(d_h)*math.log10(1+abs(d_h))\
                + (40.67 - 4.57*abs(d_h)*math.log10(1+abs(d_h)))*math.log10(R_k)\
                + 20*math.log10(d_h/7.8)+10*math.log10(20/r_h)
    return path_loss
\end{minted}
\subsection{Zabudowa wysoka}
W przypadku zabudowy wysokiej analogicznie do zabudowy niskiej trasy schodkowe i poprzeczne dają zbliżone wyniki więc zastosowałem jedną funkcję dla obu przypadków.
\begin{minted}{python3}
    def high(f_G: float , R_k: float,  h_t: float) -> float:
    path_loss = 143.21+29.74*math.log10(f_G)\
                - 0.99*math.log10(h_t)+ (47.23+3.72*math.log10(h_t))*math.log10(R_k)
    return path_loss
\end{minted}
Jednakże powoduje to zwiększe błędu oszacowania
\newline
W przypadku trasy bocznej dla zabudowy wysokiej zastosowałem funkcję:
\begin{minted}{python3}
    def high_sideways(f_G: float , R_k: float, h_t: float) -> float:
    path_loss = 135.41+12.49*math.log10(f_G)\
        - 4.99*math.log10(h_t)+ (46.84-2.34*math.log10(h_t))*math.log10(R_k)
    return path_loss
\end{minted}
\subsection{Bezpośrednia widoczność}
W tym przypadku zastosowałem jedną funkcję z zastosowaniem poleceń `if'
\begin{minted}{python3}
    def direct(f_G: float, R_k:float , r_h:float , h_t:float, l:float) -> float:
    R_bk = (4*h_t*r_h)/1000*l
    if R_k < R_bk:
        path_loss = 81.14+39.40*math.log10(f_G)\
        - 0.09*math.log10(h_t)+(15.80-5.73*math.log10(h_t))*math.log10(R_k)
        return path_loss
    if R_k > R_bk:
        path_loss = (48.38-32.10*math.log10(R_bk))+45.70*math.log10(f_G)\
        + (25.32-13.90*math.log10(R_bk))*math.log10(h_t)\
        + (32.10+13.90*math.log10(h_t))*math.log10(R_k)\
        +20*math.log10(1.6/r_h)
        return path_loss
\end{minted}
W przypadku odcinka dalekiego należy zastosować współczynnik korekcji 
\[ {\left(\Delta P_L\right)}_{ h_m} = 20\log{(1.6/h_m)} \]
w przypadku uwzględnionym w dokumencie wysokość odniesienia wnosi 1.6 metra

\end{document}